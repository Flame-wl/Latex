%告知 CTeX 这个文件是用 UTF-8 编码
% !Mode:: "TeX:UTF-8"

%使用 pdflatex
%!TEX program = pdflatex


\documentclass[12pt,a4paper]{article}
%正文字体大小为12pt, 页面规格是A4, 使用article文档格式
\usepackage[utf8]{inputenc}
%作用是inputenc用来识别输入编码
\usepackage[left=2.2cm,right=2.2cm,top=3cm,bottom=2.5cm]{geometry}
%latex设置页面边距,页面大小,页边距
\usepackage{mathtools}
%数学公式扩展宏包,提供了公式编号定制和更多的符号、矩阵等
\usepackage{booktabs}  
%booktabs宏包画三线表,线条精细可变
\usepackage{graphicx} 
%支持插图
\usepackage{listings}
%提供了排版关键字高亮的代码环境 lslisting 以及对版式的自定义。
%类似宏包有minted
\usepackage{amssymb}
%打出因为所以那三个点的宏包, 导言区宏包

\lstset{%使用\lstset{}进行代码环境的设置
backgroundcolor=\color{cyan!10},%% 选择代码背景,必须加上\ usepackage {color}或\ usepackage {xcolor}.
basicstyle=\ttfamily, % 设置代码字号.
numbers=left,% 给代码添加行号,可取值none, left, right.
numberstyle=\scriptsize %小六号 \scriptsize
% numberstyle=\tiny\color{mygray}, % 行号的字号和颜色
}

\usepackage{fancyhdr}
%修改页眉页脚格式,令页眉页脚可以左对齐、居中、右对齐

\usepackage{tikz}%以 TikZ 为基础提供排版样式丰富的彩色盒子的功能
\usepackage[europeanresistors,americaninductors]{circuitikz}
%欧式电阻,美式电感

\usepackage{indentfirst} %令章节标题后的第一段首行缩进
\usepackage[wby]{callouts}

\lstdefinestyle{mystyle}{
    backgroundcolor=\color{white},  %背景颜色 
    commentstyle=\color{codegreen}, %注释风格
    keywordstyle=\color{magenta},   %关键字风格 红紫色
    numberstyle=\tiny\color{codegray},
    stringstyle=\color{codepurple},
    basicstyle=\footnotesize\ttfamily,breaklines=true,
    %设置代码的大小  选择一种等宽(“打字机”)字体族  对过长的代码自动换行  
    breakatwhitespace=false, %空格中断
    xleftmargin=20pt, %x左边框
    xrightmargin=20pt,%x右边框       
    breaklines=true,  %代码过长则换行               
    captionpos=b,     % 设置标题位置.               
    keepspaces=true,  % 保留空格    有助于保持代码的缩进 possibly needs columns=flexible           
    numbers=left,     % 给代码添加行号,可取值none, left, right.                   
    numbersep=5pt,    % 设置行号与代码之间的间隔              
    showspaces=false, % 显示每个地方添加特定下划线的空格; 覆盖了'showtringspaces'               
    showstringspaces=false, % 仅在字符串中允许空格
    showtabs=false,   % 在字符串中显示添加特定下划线的制表符               
    tabsize=2,        % 将默认tab设置为2个空格  
    framextopmargin=50pt,%代码区定框
    frame=bottomline,   %代码区底部
    basicstyle=\footnotesize\ttfamily,  % 设置代码字号
    language=Octave     % 使用的语言
}
\usepackage{ulem}%提供排版可断行下划线的命令 \uline 以及其它装饰文字的命令
\lstset{style=mystyle} %代码环境设置  自定义版式,将mystyle中版式导入
\linespread{1.5}       %行距1.5倍 
\title{\textbf{\texttt{Electric Circuits - Homework 02}}}
\author{Automation Class 1904 - 19401100112}
\pagestyle{fancy}   %使用fancy风格
\fancyhf{} % 清空当前设置
\rhead{HW2 Edition} %页眉右边
\rfoot{fireowl}       %页脚右边
\lhead{Electric Circuits} % 页眉左边
\cfoot{\thepage}    %页脚中间 页码
\thispagestyle{plain}
% empty
% 无页眉页脚
% plain
% 无页眉,页脚为居中页码
% headings
% 页眉为章节标题,无页脚
% myheadings
% 页眉内容可自定义,无页脚
\date{(Due date: 2020/10/17)}%自定义日期\today显示电脑上的日期-英文版

\begin{document}
\maketitle

\begin{enumerate}
	
    \item (10 Point)\\
    \begin{quote}
        Ans:\\
        \begin{center}
            \begin{circuitikz}[american]
                \draw (0,0) to[V=$120V$,invert] (0,3) %invert使得方向相反
                to[R, R = $4\Omega$,-*] (2,3)
                to[short, -*] (4,3) to [R,R=$3\Omega$](6,3)
                to[R,R=$9\Omega$](6,0)
                to[short, -*] (4,0)
                to[short,-*] (2,0) --(0,0);
                \draw (4,3) to[R= $18\Omega$] (4,0);
                \draw [->,-latex, blue] (0.6,2.6) to (1.6,2.6);
                \node at (1.1,2.2)[blue]{$i_s$};
                \draw [->,-latex, blue] (3.6,2) to node [left, blue] {$i_1$} (3.6,1);
                \draw [->,-latex, blue] (5.6,2) to node [left, blue] {$i_2$} (5.6,1);
                \node at (2,3.3){x};
                \node at (2,-0.3){y};
            \end{circuitikz}
        \end{center}
        a).Since in the circuit the sum of the resistance is equivalent to
        \begin{center}
            $R =$4 + $(\frac{1}{18} + \frac{1}{12})$$^{-1}$ = $11.2 \Omega$
        \end{center}
    	\quad Thus using Ohm's law we compute the value of $i_s$ and $u_1$:
    	\begin{center}
    		i$_s$ = $\frac{120}{R}$  = $\frac{75}{7}$ A ; 
    		\qquad
    		$u_1$ = $(\frac{1}{18} + \frac{1}{12})$$^{-1}$ $\times$ $i_s$ = $\frac{540}{7}V$
    	\end{center}
    	\quad Then using Ohm's too:
    	\begin{center}
    		$i_1$ = $u_1$ / 18 = $\frac{30}{7}$A;
    		\qquad 
    		$i_2$ = $u_1$ / 12= $\frac{45}{7}$A
    	\end{center}
    	\quad Last, we can use P = i$^2$R compute the value of each resistor:
    	\begin{center}
    		P$_{18\Omega}$ = $\frac{16200}{49}$W; \qquad P$_{9\Omega}$ = $\frac{18225}{49}$W
    		\\
    		P$_{3\Omega}$ = $\frac{6075}{45}$W; \qquad P$_{4\Omega}$ = $\frac{22500}{49}$W
    	\end{center}
    	b).Given the current $i_s$, we can figure it out the power by the 120V source:
    	\begin{center}
    		P$_{120V}$ = 120$\times$$i_s$ = $\frac{9000}{7}$W
    	\end{center}
    	c).The power delivered equals the power dissipated is:
    	\begin{center}
    		P$_{diss}$ = $\frac{16200}{49}$ + $\frac{18225}{49}$ + $\frac{6075}{45}$ + $\frac{22500}{49}$ = $\frac{9000}{7}$W
    	\end{center}
    \end{quote}

	\item(10 Point)
		\begin{quote}
			Ans:\\
			\begin{center}
				\begin{circuitikz}[american]
					\draw (0,0) to[V=$900V$,invert] (0,3) %invert使得方向相反
					to [R,l = $30\Omega$, -*] (2.5,3)to (2.5,4.5) 
					to [R,l=$30\Omega$, -*] (5,4.5)
					to [R,l=$182\Omega$](7.5,4.5) to [short,-*]  (7.5,3) to (8.5,3)%无元件画带点线用short
					to [R,l=$17\Omega$, ] (8.5,0) to (0,0);
					\draw (2.5,3) to(2.5,1.5)
					to [R, l_=$10\Omega$, -*] (5,1.5)
					to [R,l_=$44\Omega$,] (7.5,1.5) to  (7.5,3) ;
					\draw (5,4.5) to[R,l=$60$] (5,1.5);
					\draw [->, -latex, blue] (3.3,4) to (4.3,4);
					\node at (3.8,3.6)[blue]{$i_1$};
					\node at (4.6,4.2){$+$};
					\node at (4.6,3){$v$};
					\node at (4.6,1.8){$-$};
					\draw [->, -latex, blue] (5.8,1.9) to (6.8,1.9);
					\node at (6.3,2.2) [blue]{$i_2$};
				\end{circuitikz}
			\end{center}
		
			\quad Replace the 10-30-60	delta with a wye equivalent to get
			\\
			\begin{center}
				\begin{circuitikz}[american]
					\draw (0,0) to[V=$900V$,invert] (0,3) %invert使得方向相反
					to [R,l = $30\Omega$, -*] (2,3)
					to [R,l = $3\Omega$, -*] (3.5,3)
					to [R,l=$18\Omega$] (5,4.5)
					to [R,l=$182\Omega$](7.5,4.5) to [short,-*]  (7.5,3) to (8.5,3)%无元件画带点线用short
					to [R,l=$17\Omega$, ] (8.5,0) to (0,0);
					\draw (3.5,3) to [R, l_=$6\Omega$] (5,1.5)
					to [R,l_=$44\Omega$,] (7.5,1.5) to  (7.5,3) ;
					\node at (4.6,3.6){$+$};
					\node at (4.6,3){$v$};
					\node at (4.6,2.4){$-$};
					\draw [->, -latex, blue] (5.8,1.9) to (6.8,1.9);
					\node at (6.3,2.2) [blue]{$i_2$};
				\end{circuitikz}
			\end{center}
		
		\quad a). We know the total resistance by the equivalent resistance:
		\begin{center}
			$R$ = $30+3+17$ + ($\frac{1}{18+182}$ + $\frac{1}{6+44}$)$^{-1}$ = 90$\Omega$
		\end{center}
		\qquad Then we can figure out the total current i$_s$:
		\begin{center}
			i$_s$ = $\frac{900}{90}$ = 10A
		\end{center}
		\qquad Using Ohm's law and parallel shunt to calculate the current through the (18+182)$\Omega$ resistance i$_3$ and i$_2$:
		\begin{center}
			i$_2$ = 10 $\times$ $\frac{(6+44)||(18+182)}{6+44}$ = 8A;\quad i$_3$ = 10 $\times$ $\frac{(6+44)||(18+182)}{18+182}$ = 2A
		\end{center}
		\qquad Because the voltage across the i$_1$ current is equal to the sum  of the voltages of 3$\Omega$ and 18$\Omega$. Thus we can figure out i$_1$:
		\begin{center}
			i$_1$ = $\frac{3i_s + 18i_3}{30}$ = 2.2A
		\end{center}
		
		\quad b).
		\begin{center}
			v = $18i_3 - 6i_2$ = -12V
		\end{center}
	
		\quad c).
		\begin{center}
			i$_2$ = 8A;
		\end{center}
		
		\quad d).
		\begin{center}
			P = $900i_s$ = 9000W
		\end{center}
		\end{quote}
	
	\item(15 Point)	
		\begin{quote}
			Ans:\\
		\qquad From Eq 3.69 and Eq 3.50
		\begin{center}
				$\frac{i_1}{i_3} = \frac{R_2R_3}{D}$;\qquad $\frac{R_3}{R_1} = \frac{i_1^2}{i_3^2}$
		\end{center}
		 Thus:
		\begin{center}
			$R_3 = \frac{D^2}{R_1R_2^2}$
		\end{center}
		 Then we konwn :
		\begin{center}
			D = $(R_1 + 2R_a)(R_2 + 2R_b) + 2R_2R_b$;\\
			$R_a = \sigma R_1$\qquad
			$R_b = \frac{(1+2\sigma)^2\sigma R1}{4(1+\sigma)^2}$;\qquad
			$R_2 = (1+2\sigma)^2R_1$ 
		\end{center}
		 So we can calculate D:\\
		\begin{center}
			D = $\frac{R_1^2(1+2\sigma)^4}{1+\sigma}$
		\end{center}
		 Solving for R$_3$ gives:
			\begin{center}
				$R_3 = \frac{D^2}{R_1R_2^2} = \frac{R_1(1+2\sigma)^4}{(1+\sigma)^2}$
			\end{center}
			
		\end{quote}
		
		\item(15 Point)
		\begin{quote}
			\quad Ans:\\
			\begin{center}
				\begin{circuitikz}[american]
				\draw (0,0) to [V, l=$18V$, invert] (0,3)
				to [short, -*] (2,3)
				to [R, l=$10\Omega$, -*] (4, 3)
				to [cV, l=$3v_{\Delta}$, -*] (6,3)
				to (8,3)
				to [R, l=$40\Omega$] (8,0)
				to (0,0);
				
				\draw (2,3) to [R, l=$33\Omega$, -*] (2,0);
				\draw (4,3) to [R, l=$2\Omega$, -*] (4,0);
				\draw (6,3) to [R, l=$20\Omega$, -*] (6,0);
				
				\node at (3.5,2.5) {$+$};
				\node at (3.5,0.4) {$-$};
				\node at (3.5,1.5) {$v_\Delta$};
				
				\node at (7.5,2.5) {$+$};
				\node at (7.5,0.4) {$-$};
				\node at (7.5,1.5) {$v_0$};
				\end{circuitikz}
			\end{center}
			
			\qquad Place 3v$_\Delta$ inside a supernode and use the lower node as a reference. Then:
			\begin{center}
				$\frac{v_\Delta - 18}{10} + \frac{v_\Delta}{2} + \frac{v_\Delta - 3v_\Delta}{20} + \frac{v_\Delta - 3v_\Delta}{40} =0 $
			\end{center}
			\begin{center}
				$18v_\Delta = 72$ ; \qquad $v_\Delta = 4V$
			\end{center}
			Thus we can calculate v$_0$:
			\begin{center}
				$v_0 = v_\Delta - 3v_\Delta = -8V$
			\end{center}	
		\end{quote}
	\item (20 Point)
		\begin{quote}
			\quad Ans:\\
			\begin{center}
				\begin{circuitikz}[american]
					\draw (0,0) to [V, l=$225V$, invert, -*] (0,3)
					to (0,5) to [R, l=$5\Omega$] (8,5) to (8,3)
					to [cV, l=$10i_{\sigma}$] (8,0)
					to[R=$1\Omega$, -*] (4,0)
					to[R=$2\Omega$] (0,0);
					
					\draw (0,3) to [R=$3\Omega$, -*] (4,3)
					to[R=$4\Omega$, -*] (8,3);
					
					\draw (4,3) to [R, l=$20\Omega$] (4,0);
					 
					\draw  (2.5, 2.6)[->, -latex, blue]  to (1.5,2.6);
					\node at (2, 2.3) [blue] {$i\sigma$};
					
					\draw (2.8,1.3)  [<-]arc(0:270:0.8);    %!!! 这个画曲线箭头非常可
					\node at (2.1,1.3) [blue ]{$i_1$};
					
					\draw (4.7,3.9)  [<-]arc(0:270:0.8);    %!!! 这个画曲线箭头非常可
					\node at (4,3.9) [blue ]{$i_2$};
					
					\draw (6.8,1.3)  [<-]arc(0:270:0.8);    %!!! 这个画曲线箭头非常可
					\node at (6.1,1.3) [blue ]{$i_3$};
					
					
				\end{circuitikz}
			\end{center}
		    Through the mesh-current method we can get:
			
			\qquad	$-255 + 3(i_1-i_3) + 20(i_1-i_2) + 2i_1 = 0$ \qquad (mesh 1)
			
			
			
			\qquad	$5i_2+4(i_2-i_3)+3(i_2-i_1) = 0$	\qquad  (mesh 2)
			
			
			\qquad $4(i_3-i_2)+10i_\sigma+i_3+20(i_3-i_1) = 0$ \qquad  (mesh 3)
			
		
			
			\qquad	$i_2-i_1=i_\sigma$

		
			 simplified as:

			
			\qquad	$25i_1 - 3i_2 -20i_3 + 0i_\sigma= 225$ \qquad (mesh 1)
			
			
			\qquad	$-3i_1 + 12i_2-4i_3+0i_\sigma = 0$		\qquad (mesh 2)
			
			
			\qquad	$-20i_1-4i_2+25i_3+10i_\sigma = 0$		\qquad (mesh 3)
			
			
			\qquad	$i_1 - i_2 + 0\times i_3 +i_\sigma = 0$\\
		    Solving:
		    \begin{center}
		    	$i_1 = 108A$ \quad $i_2 = 65A$ \quad $i_3 = 114A$ \quad $i_\sigma = -43A$
		    \end{center}
		
			Thus the power dissipate in the 20$\Omega$ resistor in the circuit is:
			\begin{center}
				$P_{20\Omega}$ = $(i_3 - i_1)^2 \times 20 = 720W$
			\end{center}
			And the power extracted by the current con-trolled voltage source is:
			\begin{center}
				$P_{10i_\sigma} = 10i_\sigma i_3 = -49020W$
			\end{center}
			\end{quote}
	\item (10 Points)
		\begin{quote}
			\quad Ans:\\
			\begin{center}
				\begin{circuitikz}[american]
					\draw (0,0) to [I, l=$160A$] (0,3)
					to (0,5) to [R, l=$1\Omega$, f=$i_\Delta$] (8, 5)
					to [short, -*] (8,3)
					to [cV, l = 8$i_\Delta$] (8,0) to (0,0);
					
					\draw (0,3) to [R, l=$5\Omega$, *-*] (4,3)
					to [R, l=$4\Omega$, -*] (8,3);
					
					\draw (4,3) to [R, l=$20\Omega$, -*] (4,0);
					
					\draw (2.8,1.5)  [<-]arc(0:270:0.8);    %!!! 这个画曲线箭头非常可
					\node at (2,1.5) [blue ]{$i_1$};
					
					\draw (4.8,3.9)  [<-]arc(0:270:0.8);    %!!! 这个画曲线箭头非常可
					\node at (4,4) [blue ]{$i_2$};
					
					\draw (6.8,1.5)  [<-]arc(0:270:0.8);    %!!! 这个画曲线箭头非常可
					\node at (6,1.5) [blue ]{$i_3$};					
				\end{circuitikz}
			\end{center}
		
		
		Through the mesh-current method and {$i_1 = 160A , i_2 = i_\Delta$} we can get:
		
		\begin{center}
			$4(i_3 - i_\Delta) + 8i_\Delta + 20(i_3 - 160) = 0$
		
		
			$i_\Delta + 4(i_\Delta - i_3) + 5(i_\Delta - 160) = 0$\\	
		\end{center}
		
		simplified as:
		\begin{center}
			$24i_3 + 4i_\Delta = 3200$
			
			
			$10i_\Delta-4i_3 = 800$\\
		\end{center}
		
		Solving:
		\begin{center}
			$i_\Delta = 125A$; \qquad $i_3 = 112.5A$
		\end{center}

		\quad According to the relationship between node voltages:
		\begin{center}
			$v_{160A} = 8i_\Delta + i_\Delta = 1125V$
		\end{center}
		Then :\\
		\begin{center}
			$P_{160A} = -160v_{160A} = -180000W$ ; \qquad $P_{8i_\Delta} = 8i_\Delta i_3 = 112500W$
		\end{center}
		\quad Therefore, the independent source is developing 180000 W, all other elements are absorbing power, and the total power developed is thus 180000W.\\
		Let's Check it :
		\begin{center}
			$P_{1\Omega} = i_\Delta^2 \times 1 = 15625W$; \qquad $P_{5\Omega} = (i_1 - i_\Delta)^2 \times 5 = 6125W$\\
			$P_{4\Omega} = (i_\Delta - i_3)^2 \times 4 = 625W$; \qquad $P_{20\Omega} = (i_1 - i_3)^2 \times 20= 45125W$
		\end{center}
		\begin{center}
			$\Sigma_{P} = 112500 + 15625+6125+625+45125 = 180000W$ (CHECKS)
		\end{center}
		\end{quote}
	\item (10 Points)
	\begin{quote}
		\quad Ans:\\
		\begin{center}
			\begin{circuitikz}[american]
				\draw (0,0) to [V, l=$240V$,\begin{center}
					\begin{circuitikz}[american]
						\draw (0,0) to [V, l=$60V$,-*,invert] (0,3) 
						to [R, l=$10\Omega$] (3,3) 
						to [R, l= $8\Omega$,*-*] (6,3)
						to [short,-*] (8,3);
						\draw (3,3) to[R, l=$40\Omega$, *-*] (3,0);
						\draw(0,0) to[short,-*] (8,0);
						\draw (0,3) to (0,5)
						to [I, l= $6A$] (6,5)
						to (6,3);
						\node at (8,3.2)[blue] {a};
						\node at (8,-0.3)[blue] {b};
					\end{circuitikz}
				\end{center}invert] (0,3)
				to [R, l = $12\Omega$, -*] (4,3)
				to [R, l=$15\Omega$, -*] (8,3)
				to (12,3) to [dcisource, l= $i_{dc}$] (12,0)
				to[short, -*] (8,0)
				to [R, l=$40\Omega$, -*] (4,0) to(0,0);
				
				\draw (4,3) to[R, l=$20\Omega$] (4,0);
				\draw (8,3) to[R, l=$50\Omega$] (8,0);
				
				\draw (4,3) to (4,5) to[I, l=$10A$ , invert] (8,5) to(8,3);	
				
				\node at (3.8, 3.2) [blue] {1};
				\node at (8.6, 3.2) [blue] {v= 0};
				\node at (3.8, 0.2) [blue] {2};
				\node at (8.2, 0.2) [blue] {3};
			\end{circuitikz}
		\end{center}
		Since the 10 A sourve is developing 0 W, v$_1$ must be 0 V.\\
		Since V$_1$ is known, we can sum the currents away from node 1 to find $v_2$; Thus:
		\begin{center}
			$\frac{0-(240+v_2)}{12} + \frac{0-v_2}{20} + \frac{0}{15} - 10 = 0$
			\qquad$\therefore v_2 = -225V $
		\end{center}
		Now that we know $v_2$ we sum the current away form node to find $v_3$;\\ Thus:
		\begin{center}
			$\frac{v_2+240-0}{12} + \frac{v_2-0}{20} + \frac{v_2-v_3}{40} = 0 $
			\qquad$\therefore v_3 = -625V $
		\end{center}
		Now that we know $v_3 we sum the currents away from node 3 to find i_{dc} ;$\\ Thus:
		\begin{center}
			$\frac{v_3}{50} + \frac{v_3 - v_2}{40} - i_{dc} = 0$ \qquad $\therefore i_{dc} = -22.5A$
		\end{center}
	\end{quote}

	\item(10 Points)
	\begin{quote}
		\quad Ans:\\
		\qquad [a] Apply source transformations to both current sources to get:
		\begin{center}
			\begin{circuitikz}[american]
				\draw (0,0) to[V, l=$5.4V$, invert] (0,2)
				to [R,l=$2.7k\Omega$, -*] (3,2) 
				to [R, l=$2.3k\Omega$, -*] (6,2)
				to[R, l=$1k\Omega$] (9,2)
				to [V, l=$1.2V$, invert] (9,0) to (0,0);
				
				\draw (5,1.6) [->, -latex, blue] to (4,1.6);
				\node at (4.5, 1.3) [blue]{$i_0$};
			\end{circuitikz}
		\end{center}
		\begin{center}
			$i_0 = \frac{-(5.4+1.2)}{2700+2300+1000}= -1.1mA$
		\end{center}
		\qquad [b]
			\begin{center}
				\begin{circuitikz}[american]
					\draw (0,0) to[I, l=$2mA$] (0,2)
					to  (3,2) 
					to [R, l=$2.3k\Omega$, *-*] (6,2)
					to (9,2)
					to [I, l=$1.2mA$] (9,0) to (0,0);
					
					\draw (5,1.6) [->, -latex, blue] to (4,1.6);
					\node at (4.5, 1.3) [blue]{$i_0$};
					
					\draw (3,2) to [R,l=$2.7k\Omega$, -*] (3,0);
					\draw (6,2) to [R, l=$1k\Omega$,-*] (6,0);
					
					\draw (3,0) to[short] node [ground] {GND}(3,-1);
					
					\node at (2.8,2.2){$v_1$};
					\node at (6.2, 2.2) {$v_2$};
				\end{circuitikz}
			\end{center}
		\qquad The node voltage equations:
		\begin{center}
			$-2 \times 10^{-3} + \frac{v_1}{2700} + \frac{v_1-v_2}{2300} = 0$
		\end{center}
		\begin{center}
			$\frac{v_2}{1000} + \frac{v_2 - v_1}{2300} + 1.2 \times 10^{-3} = 0$
		\end{center}
		\qquad simplified as:
		\begin{center}
			$50v_1 - 27v_2 = 124.2$
		\end{center}
		\begin{center}
			$-10v_1 + 33v_2 = -27.6$
		\end{center}
		\qquad Solving:
		\begin{center}
			$v_1 = 2.43V$ \qquad $v_2 = -0.1V$
		\end{center}
	\begin{center}
		$\therefore i_0 = \frac{v_2 - v_1}{2300}= -1.1mA$ 
	\end{center}
		
	\end{quote}



\end{enumerate}








\end{document}