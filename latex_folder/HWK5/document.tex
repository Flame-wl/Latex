%告知 CTeX 这个文件是用 UTF-8 编码
% !Mode:: "TeX:UTF-8"

%使用 pdflatex
%!TEX program = pdflatex


\documentclass[12pt,a4paper]{article}
%正文字体大小为12pt, 页面规格是A4, 使用article文档格式
\usepackage[utf8]{inputenc}
%作用是inputenc用来识别输入编码


\usepackage[left=2.2cm,right=2.2cm,top=3cm,bottom=2.5cm]{geometry}
%latex设置页面边距,页面大小,页边距
\usepackage{mathtools}
%数学公式扩展宏包,提供了公式编号定制和更多的符号、矩阵等
\usepackage{booktabs}  
%booktabs宏包画三线表,线条精细可变
\usepackage{graphicx} 
%支持插图
\usepackage{listings}
%提供了排版关键字高亮的代码环境 lslisting 以及对版式的自定义。
%类似宏包有minted
\usepackage{amssymb}
%打出因为所以那三个点的宏包, 导言区宏包

\lstset{%使用\lstset{}进行代码环境的设置
	backgroundcolor=\color{cyan!10},%% 选择代码背景,必须加上\ usepackage {color}或\ usepackage {xcolor}.
	basicstyle=\ttfamily, % 设置代码字号.
	numbers=left,% 给代码添加行号,可取值none, left, right.
	numberstyle=\scriptsize %小六号 \scriptsize
	% numberstyle=\tiny\color{mygray}, % 行号的字号和颜色
}

\usepackage{fancyhdr}
%修改页眉页脚格式,令页眉页脚可以左对齐、居中、右对齐

\usepackage{tikz}%以 TikZ 为基础提供排版样式丰富的彩色盒子的功能
\usepackage[europeanresistors,americaninductors]{circuitikz}
%欧式电阻,美式电感

\usepackage{indentfirst} %令章节标题后的第一段首行缩进
\usepackage[wby]{callouts}

\lstdefinestyle{mystyle}{
	backgroundcolor=\color{white},  %背景颜色 
	commentstyle=\color{codegreen}, %注释风格
	keywordstyle=\color{magenta},   %关键字风格 红紫色
	numberstyle=\tiny\color{codegray},
	stringstyle=\color{codepurple},
	basicstyle=\footnotesize\ttfamily,breaklines=true,
	%设置代码的大小  选择一种等宽(“打字机”)字体族  对过长的代码自动换行  
	breakatwhitespace=false, %空格中断
	xleftmargin=20pt, %x左边框
	xrightmargin=20pt,%x右边框       
	breaklines=true,  %代码过长则换行               
	captionpos=b,     % 设置标题位置.               
	keepspaces=true,  % 保留空格    有助于保持代码的缩进 possibly needs columns=flexible           
	numbers=left,     % 给代码添加行号,可取值none, left, right.                   
	numbersep=5pt,    % 设置行号与代码之间的间隔              
	showspaces=false, % 显示每个地方添加特定下划线的空格; 覆盖了'showtringspaces'               
	showstringspaces=false, % 仅在字符串中允许空格
	showtabs=false,   % 在字符串中显示添加特定下划线的制表符               
	tabsize=2,        % 将默认tab设置为2个空格  
	framextopmargin=50pt,%代码区定框
	frame=bottomline,   %代码区底部
	basicstyle=\footnotesize\ttfamily,  % 设置代码字号
	language=Octave     % 使用的语言
}
\usepackage{ulem}%提供排版可断行下划线的命令 \uline 以及其它装饰文字的命令
\lstset{style=mystyle} %代码环境设置  自定义版式,将mystyle中版式导入
\linespread{1.5}       %行距1.5倍 
\title{\textbf{\texttt{Electric Circuits - Homework 05}}}
\author{Automation Class 1904}
\pagestyle{fancy}   %使用fancy风格
\fancyhf{} % 清空当前设置
\rhead{HW3 Edition} %页眉右边
\rfoot{fireowl}       %页脚右边
\lhead{Electric Circuits} % 页眉左边
\cfoot{\thepage}    %页脚中间 页码
\thispagestyle{plain}
% empty
% 无页眉页脚
% plain
% 无页眉,页脚为居中页码
% headings
% 页眉为章节标题,无页脚
% myheadings
% 页眉内容可自定义,无页脚
\date{(Due date: 2020/1/3)}%自定义日期\today显示电脑上的日期-英文版

\begin{document}
	\maketitle
	
	\begin{enumerate}	
		\item (10 Point)
			Ans:
		\begin{quote}
			a).By the question:
			\begin{center}
				$f=400Hz$
			\end{center}
			b).From this question we know:
			\begin{center}
				$\theta_v = 0^{\circ}$\\
				$I=\frac{V}{j\omega L} = \frac{100}{\omega L}\angle-90^{\circ}$
			\end{center}
			Thus:
			\begin{center}
				$\theta_i = -90^{\circ}$
			\end{center}
			c).By the question b and condition:
			\begin{center}
				$I_m=\frac{100}{\omega L} = 20$
			\end{center}
			Thus:
			\begin{center}
				$\omega L = 5\Omega$
			\end{center}
			d).Because:
			\begin{center}
				$\omega = 2\pi f=800\pi$
			\end{center}
			Then:
			\begin{center}
				$L = \frac{5}{800\pi}=1.99mH$
			\end{center}
			e).
			\begin{center}
				$Z_L=j\omega L=j5\Omega$
			\end{center}
		\end{quote}
		\clearpage %另起一页
		
		\item (10 Points)Ans:
		\begin{quote}
			By the question use Node-voltage Method:
			\begin{center}
				$\frac{v_1-v_g}{20}+\frac{v_1}{j2}+\frac{v_1-v_2}{Z} = 0$\\
				$\frac{v_2-v_1}{Z}+\frac{v_2}{-j10}+\frac{v_2-v_g}{3+j1}-I_g=0$
			\end{center}
			Thus 
			\begin{center}
				$v_2=209-j63 V$
			\end{center}
			Use $v_2$ to equotion 1:
			\begin{center}
				$Z=-0.9+j6.37$
			\end{center}
		\end{quote}
		\clearpage
		
		\item (20 Points)Ans:
		\begin{quote}
		Step 1: Voltage source to Current source:
		\begin{center}
			$I_1 = \frac{240\angle 0^\circ}{j60-j36} = 10\angle -90^\circ A$
		\end{center}
		Step2: Calculate impedance Z
		\begin{center}
			$Z=j24||24=12+j12$
		\end{center}
		Step3: Current source to Voltage source
		\begin{center}
			$V_2=I_1Z=120-j120=120\sqrt{2}\angle -45^\circ V$
		\end{center}
		\end{quote}
		\clearpage
		
		\item (10 Points)
		\begin{quote}
			Ans:\\
			By the question use Node-Voltage Method:
			\begin{center}
				$\frac{v_0}{-j8}+\frac{v_0-2.4I_\Delta}{j4}+\frac{v_0}{5}-(10+j20)=0$\\
				$I_\Delta = \frac{v_0}{-j8}$
			\end{center}
			Simplify it:
			\begin{center}
				$I_\Delta-2I_\Delta+j0.6I_\Delta-j1.6I_\Delta=10+j20$\\
				$I_\Delta=-15-j5$
			\end{center}
			Thus:
			\begin{center}
				$v_0=-40+j120$
			\end{center}
		\end{quote}
		
		\clearpage
		\item (10 Points)
		\begin{quote}
			Ans:\\
			By the question:
			\begin{center}
				$Z_{Th}=45+j125+(\frac{\omega M}{|Z_{22}|})^2\bar Z_{22}=77+j109$
			\end{center}
			Then:
			\begin{center}
				$V_{Th} = \frac{425}{10+j5}\times j20=340+j680$
			\end{center}
		\end{quote}
		\clearpage
		
		\item (10 Points)
		\begin{quote}
			Ans:\\
			By the question:
			\begin{center}
				$i_g=4cos2000t \quad mA = 4\angle0^\circ mA$
			\end{center}
			Then:
			\begin{center}
				$Z_c=\frac{1}{j\omega c}=-j3125\Omega$\\
				$Z_L=j\omega L=-j200\Omega$\\
				$Z_{eq}=500+[-j3125||(j200+1000)]=\frac{23269500}{15289}-j\frac{2075000}{15289}$
			\end{center}
		 Thus:
		\begin{center}
			$P_g=-\frac{1}{2}|I|^2Re{Z_{eq}}=-12mW$
		\end{center}
		The source delivers 12mW of power to the circuit.
		\end{quote}
	
		\clearpage
		\item (10 Points)
		\begin{quote}
			Ans:\\
			a).By the question:
			\begin{center}
				$S_1=16+j28  \quad KVA$\\
				$S_2=6-j8  \quad KVA$\\
				$S_3=8+j0 \quad KVA$
			\end{center}
		Then:
		\begin{center}
			$S=S_1+S_2+S_3=30+j20$
		\end{center}
		And:
		\begin{center}
			$200I^*=S$\\
			$I=150-j100$
		\end{center}
	 	Then:
	 	\begin{center}
	 		$Z=\frac{200}{150-j100}=0.923+j0.615=1.11\angle33.69^\circ$
	 	\end{center}
 		b).By a:
 		\begin{center}
 			$pf=cos(33.69^\circ)=0.8321 \quad lagging$
 		\end{center}	
		\end{quote}
		
		\clearpage
		\item (10 Points)
		\begin{quote}
			Ans:\\
			a).Open circuit:
			\begin{center}
				$v_{Th}=\frac{760}{28+j96}(j50)=364.8+j106.4$
			\end{center}
			Short circuit:
			\begin{center}
				$(28+j96)I_1-j50I_{sc}=120$\\
				$-j50I_1-(31+j100)I_{sc}=0$
			\end{center}
			Thus:
			\begin{center}
				$I_{sc}=0.48-j0.518$\\
				$|I_{sc}|=0.7A$
			\end{center}
			Then:
			\begin{center}
				$Z_{Th}=\frac{V_{Th}}{I_{sc}}=521+j152$\\
				$Z_L=Z_{Th}^*=521-j152$\\
				$I_L=\frac{v_{Th}}{Z_{Th}+Z_L}=0.35+j0.1$
				$P_L=|I_L|^2Re(Z_L)=69W$
			\end{center}
			b).By the question:
			\begin{center}
				$I_1=\frac{Z_{22}I_2}{j\omega M}=0.762-j0.0017=0.762\angle 1.28^\circ$
			\end{center}
		And:
		\begin{center}
			$P_{tr}=760\times 0.762\times cos1.28 -(0.762)^2\times 8=161.4W$\\
			$\% delivered=\frac{69}{161.4}\times 100= 42.75\%$
		\end{center}
		\end{quote}
	
		\clearpage
		\item (10 Points)
		\begin{quote}
			Ans:\\
			a).By the question:
			\begin{center}
				$I_{aA}=\frac{200}{25}=8A\quad (rms)$\\
				$I_{bB}=\frac{200\angle-120^\circ}{30-j40}=4\angle -66.87^\circ A\quad (rms)$\\
				$I_{cC}=\frac{200\angle120^\circ}{80-j60}=2\angle 83.13^\circ A\quad (rms)$
			\end{center}
			The magnitudes are unequal and the phase angles are not $120^\circ apart.$\\
			b).\begin{center}
				$I_0=I_{aA}+I_{bB}+I_{cC}=9.96\angle-9.79^\circ A\quad (rms)$
			\end{center}
		\end{quote}


	\end{enumerate}
\end{document}