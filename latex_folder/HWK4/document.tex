%告知 CTeX 这个文件是用 UTF-8 编码
% !Mode:: "TeX:UTF-8"

%使用 pdflatex
%!TEX program = pdflatex


\documentclass[12pt,a4paper]{article}
%正文字体大小为12pt, 页面规格是A4, 使用article文档格式
\usepackage[utf8]{inputenc}
%作用是inputenc用来识别输入编码


\usepackage[left=2.2cm,right=2.2cm,top=3cm,bottom=2.5cm]{geometry}
%latex设置页面边距,页面大小,页边距
\usepackage{mathtools}
%数学公式扩展宏包,提供了公式编号定制和更多的符号、矩阵等
\usepackage{booktabs}  
%booktabs宏包画三线表,线条精细可变
\usepackage{graphicx} 
%支持插图
\usepackage{listings}
%提供了排版关键字高亮的代码环境 lslisting 以及对版式的自定义。
%类似宏包有minted
\usepackage{amssymb}
%打出因为所以那三个点的宏包, 导言区宏包

\lstset{%使用\lstset{}进行代码环境的设置
	backgroundcolor=\color{cyan!10},%% 选择代码背景,必须加上\ usepackage {color}或\ usepackage {xcolor}.
	basicstyle=\ttfamily, % 设置代码字号.
	numbers=left,% 给代码添加行号,可取值none, left, right.
	numberstyle=\scriptsize %小六号 \scriptsize
	% numberstyle=\tiny\color{mygray}, % 行号的字号和颜色
}

\usepackage{fancyhdr}
%修改页眉页脚格式,令页眉页脚可以左对齐、居中、右对齐

\usepackage{tikz}%以 TikZ 为基础提供排版样式丰富的彩色盒子的功能
\usepackage[europeanresistors,americaninductors]{circuitikz}
%欧式电阻,美式电感

\usepackage{indentfirst} %令章节标题后的第一段首行缩进
\usepackage[wby]{callouts}

\lstdefinestyle{mystyle}{
	backgroundcolor=\color{white},  %背景颜色 
	commentstyle=\color{codegreen}, %注释风格
	keywordstyle=\color{magenta},   %关键字风格 红紫色
	numberstyle=\tiny\color{codegray},
	stringstyle=\color{codepurple},
	basicstyle=\footnotesize\ttfamily,breaklines=true,
	%设置代码的大小  选择一种等宽(“打字机”)字体族  对过长的代码自动换行  
	breakatwhitespace=false, %空格中断
	xleftmargin=20pt, %x左边框
	xrightmargin=20pt,%x右边框       
	breaklines=true,  %代码过长则换行               
	captionpos=b,     % 设置标题位置.               
	keepspaces=true,  % 保留空格    有助于保持代码的缩进 possibly needs columns=flexible           
	numbers=left,     % 给代码添加行号,可取值none, left, right.                   
	numbersep=5pt,    % 设置行号与代码之间的间隔              
	showspaces=false, % 显示每个地方添加特定下划线的空格; 覆盖了'showtringspaces'               
	showstringspaces=false, % 仅在字符串中允许空格
	showtabs=false,   % 在字符串中显示添加特定下划线的制表符               
	tabsize=2,        % 将默认tab设置为2个空格  
	framextopmargin=50pt,%代码区定框
	frame=bottomline,   %代码区底部
	basicstyle=\footnotesize\ttfamily,  % 设置代码字号
	language=Octave     % 使用的语言
}
\usepackage{ulem}%提供排版可断行下划线的命令 \uline 以及其它装饰文字的命令
\lstset{style=mystyle} %代码环境设置  自定义版式,将mystyle中版式导入
\linespread{1.5}       %行距1.5倍 
\title{\textbf{\texttt{Electric Circuits - Homework 04}}}
\author{Automation Class 1904}
\pagestyle{fancy}   %使用fancy风格
\fancyhf{} % 清空当前设置
\rhead{HW3 Edition} %页眉右边
\rfoot{fireowl}       %页脚右边
\lhead{Electric Circuits} % 页眉左边
\cfoot{\thepage}    %页脚中间 页码
\thispagestyle{plain}
% empty
% 无页眉页脚
% plain
% 无页眉,页脚为居中页码
% headings
% 页眉为章节标题,无页脚
% myheadings
% 页眉内容可自定义,无页脚
\date{(Due date: 2020/12/27)}%自定义日期\today显示电脑上的日期-英文版

\begin{document}
	\maketitle
	
	\begin{enumerate}	
		\item (10 Point)
			Ans:
		\begin{quote}
			a).By the question and figure:
			\\right:
			\begin{center}
				$-2\frac{di_g}{d_t} + 16\frac{di_2}{d_t}+32i_2 = 0$
			\end{center}
			\begin{center}
				$8\frac{di_2}{d_t} + 16i_2 = \frac{di_g}{d_t}$
			\end{center}
			b).From the question :
			\begin{center}
				$i_2=e^{-t}-e^{-2t}$
			\end{center}
			Then we know:
			\begin{center}
				$8\frac{di_2}{d_t} + 16i_2=-8e^{-t}+16e^{-2t}+16e^{-t}-16e^{-2t} = 8e^{-t}$
				\\$\frac{di_g}{d_t} = 8e^{-t}$
			\end{center}
			Thus, satisfied.\\
			c).By question b and a:
			\begin{center}
				$v_1 = 4\frac{di_g}{d_t}-2\frac{di_2}{d_t} = 34e^{-t}-4e^{-2t}$
			\end{center}
			d).By the question c:
			\begin{center}
				$v_1{(0)} = 30v$
			\end{center}
			Yes, it make sense.
		\end{quote}
		\clearpage %另起一页
		
		\item (10 Points)
		\begin{quote}
		a).By the question, $i_1 and i_2$ is clockwise mesh currents in the left and right.\\
		Then use Mesh-current method:
		\begin{center}
			$v_{ab} = L_1\frac{di_1-i_2}{dt}+M\frac{di_2}{dt}$\\
			$0=L_2\frac{di_2}{dt}+M\frac{di_1-i_2}{dt}-L_1\frac{di_1-i_2}{dt}-M\frac{i_2}{dt}$
		\end{center}
		Simplified:
		\begin{center}
		    $v_{ab}=L_1\frac{di_1}{dt}+(M-L_1)\frac{di_2}{dt}$\\
		    $0=(L_1-M)\frac{di_1}{dt}+(L_1+L_2-2M)\frac{di_2}{dt}$
		\end{center}
		Thus:
		\begin{center}
			$v_{ab}=(\frac{L_1+L_2-2M}{L_1L_2-M^2})\frac{di_1}{dt}$
		\end{center}
		from which we have:
		\begin{center}
			$L_{ab}=\frac{L_1L_2-M^2}{L_1+L_2-2M}$
		\end{center}
		b).If the magnetic polarity of coil 2 is reversed, the sign of M reverses,therefore
		\begin{center}
			$L_{ab}=\frac{L_1L_2-M^2}{L_1+L_2+2M}$
		\end{center}
		\end{quote}
		\clearpage
		
		\item (20 Points)Ans:
		\begin{quote}
		With no finger on the button the circuit and $i = C\frac{dv}{dt}$,then we know:
		\begin{center}
			$C_1\frac{dv-v_s}{i}+C_2\frac{dv-v_s}{t} = 0$\\
			$C_1=C_2$
		\end{center}
		Thus:
		\begin{center}
			$2C\frac{dv}{dt}$
		\end{center}
		With a finger on the button:
		\begin{center}
			$C_1\frac{dv-v_s}{i}+C_2\frac{dv-v_s}{t}+C_3\frac{dv}{dt} = 0$
		\end{center}
		Simplified:
		\begin{center}
			$(C_1+C_2+C_3)\frac{dv}{dt}+C_2\frac{dv_s}{dt}-C1\frac{dv_s}{dt}=0$\\
			$C_1=C_2=C_3$
		\end{center}
		Thus:
		\begin{center}
			$3C\frac{dv}{dt}=0$
		\end{center}
		Since,there is no change in the output voltage of this circuit.
		\end{quote}
		\clearpage
		\item (10 Points)
		\begin{quote}
			Ans:
			$t < 0$:
			\begin{center}
				$i_L=\frac{2.5v_0}{v_0} = 2.5A$
			\end{center}
			$t > 0$:Find Thevenin resistance seen by inductor:
			\begin{center}
				$i_T = 2.5A$\\
				$R_{Th}=\frac{v_T}{i_T}=0.4$
			\end{center}
			Thus:
			\begin{center}
				$\tau = \frac{L}{R} = 12.5ms$\\
				$\frac{1}{\tau} = 80$
			\end{center}
			Then:
			\begin{center}
				$i_0 = 2.5e^{-80t}A$\\
				$v_0 = L\frac{di_0}{dt} = -e^{-80}$\qquad $t\ge0$
			\end{center}
		\end{quote}
		
		\clearpage
		\item (10 Points)
		\begin{quote}
			Ans:\\
			a).By the question and figure
			\begin{center}
				$v_0(0^-) =v_0(0^+)=120V$
			\end{center}
			Make 2.5$k\Omega$ resistor the right hand to Source Transformation.we will get resistor $37.5k\Omega$ and source 150V series.
			\begin{center}
				$v_0=-150V$\\
				$R_{eq} = 50k\Omega$
			\end{center}
			then:
			\begin{center}
				$\tau=RC=1.6ms$
				\\$\frac{1}{\tau}=625$
			\end{center}
			Thus:
			\begin{center}
				$v_0=-150+(120-(-150))e^{-625t}$\\
				$v_0=-150+270e^{-625t}$
			\end{center}
			b).By the question:
			\begin{center}
				$i_0=C\frac{dv}{dt}=6.75e^{-625t}mA$
			\end{center}
			c).By the question:
			\begin{center}
				$v_g=v_0-2.5\times 10^3=-150+253.125e^{-625t}$  
			\end{center}	
			d).By the question c:
				\begin{center}
					$v_g(0) = 103.125V$
				\end{center}
		\end{quote}
		\clearpage
		
		\item (10 Points)
		\begin{quote}
			Ans:\\
			When $0\leq t \leq 200\mu s$:
			\begin{center}
				$R_{eq}=150||100=60k\Omega$\\
				$\tau =RC=200\mu s$\\
				$\frac{1}{\tau} = 5000$
			\end{center}
			Thus:
			\begin{center}
				$v_c = 300e^{-5000t}$\\
				$v_c(200\mu s) = 110.36$
			\end{center}
		when $200\mu s \leq t \leq \infty$:
		\begin{center}
			$R_{eq} = 30||60+120||40=50K\Omega$\\
			$\tau = R_{eq}C = 166.67\mu s$\\
			$\frac{1}{\tau} = 6000$
		\end{center}
		Thus:
		\begin{center}
			$v_c = 110.36e^{-6000(t-200\mu s)}V$
		\end{center}
		When t = $300\mu s$:
		\begin{center}
			$v_c(300\mu s)=110.36e^{-6000(100\mu s)}=60.57V$\\
			$i_0(300\mu s)=\frac{u_c}{R_{eq}}=1.21mA$
		\end{center}
		Then:
		\begin{center}
			$i_1=\frac{60}{90}i_0=\frac{2}{3}i_0=0.8067mA$(direction down);
			\qquad $i_2=\frac{40}{160}i_0=\frac{1}{4}i_0 = 0.3025mA$ (direction down);
		\end{center}
	 	\begin{center}
	 		$i_{mid} = i_1-i_2= \frac{5}{12}i_0=0.5mA$(direction right)
	 	\end{center}
		\end{quote}
	
		\clearpage
		\item (20 Points)
		\begin{quote}
			Ans:\\
			By the question and figure:Find Thevenin resistance seen by capacitor
			\begin{center}
				$v_T=13\times 10^4i_\Delta+(20k\Omega || 80k\Omega)i_T$\\
				$i_\Delta = -\frac{20}{100}i_T=-0.2i_T$
			\end{center}
			Then:
			\begin{center}
				$v_T=-10\times 10^3i_T$
			\end{center}
			Thus:
			\begin{center}
				$R_{Th} = \frac{v_T}{i_T}=-10k\Omega$\\
				$\tau = R_{Th}C=-0.025$\\
				$\frac{1}{\tau} = -40$
			\end{center}
			Since:
			\begin{center}
				$v_c=20e^{40t}V$
			\end{center}
			when $v_c = 20000V$
			\begin{center}
				$40t = ln1000$\\
				$t = 172.69ms$
			\end{center}
		\end{quote}
		
		\clearpage
		\item (10 Points)
		\begin{quote}
			Ans:\\
			a).By the question and figure, When $0 \leq t \leq 0.5$:
			\begin{center}
				$i = \frac{21}{60} + (\frac{30}{60}-\frac{21}{60})e^{-\frac{t}{\tau}}$\\
				$\tau = \frac{L}{R}$
			\end{center}
			Thus :
			\begin{center}
				$i=0.35+0.15e^{-\frac{-60t}{L}}$
			\end{center}
			When t = 0.5:
			\begin{center}
				$i(0.5)=0.35+0.15e^{-\frac{-30}{L}}=0.40$
			\end{center}
			So:
			\begin{center}
				$e^{-\frac{30}{L}} = 3;$\\
				$L=\frac{30}{ln3}=27.31H$	
			\end{center}
			b).Hypothesized $t_r$ is the time the relay releases:$\qquad 0 \leq t \leq t_r$
			\begin{center}
				$i=0+(\frac{30}{60}-0)e^{\frac{-60t}{L}} = 0.5e^{\frac{-60t}{L}}$
			\end{center}
			When i = 0.4, the relay released:
			\begin{center}
				$0.4=0.5e^{\frac{-60t_r}{L}}$
			\end{center}
			Thus:
			\begin{center}
				$t_r = \frac{27.31ln1.25}{60} \cong 0.1s$
			\end{center}
			 
		\end{quote}
		
		
		
		
	\end{enumerate}
\end{document}