%DO NOT EDIT THIS AUTOMATICALLY GENERATED FILE, run "make changelog" at toplevel!!!
The major changes among the different circuitikz versions are listed
here. See \url{https://github.com/circuitikz/circuitikz/commits} for a
full list of changes.

\begin{itemize}
\item
  Version 1.0.2 (2020-03-22)

  \begin{itemize}
  \tightlist
  \item
    added Schottky transistors (thanks to a suggestion by Jérôme
    Monclard on GitHub)
  \item
    fixed formatting of \texttt{CHANGELOG.md}
  \end{itemize}
\item
  Version 1.0.1 (2020-02-22)

  Minor fixes and addition to 1.0, in time to catch the freeze for
  TL2020.

  \begin{itemize}
  \tightlist
  \item
    add v1.0 version snapshots
  \item
    added crossed generic impedance (suggested by Radványi Patrik Tamás)
  \item
    added open barrier bipole (suggested by Radványi Patrik Tamás)
  \item
    added two flags to flip the direction of light's arrows on LED and
    photodiode (suggested by karlkappe on GitHub)
  \item
    added a special key to help with precision loss in case of
    fractional scaling (thanks to AndreaDiPietro92 on GitHub for
    noticing and reporting, and to Schrödinger's cat for finding a fix)
  \item
    fixed a nasty bug for the flat file generation for ConTeXt
  \end{itemize}
\item
  Version 1.0 (2020-02-04)

  And finally\ldots{} version 1.0 (2020-02-04) of \texttt{circuitikz} is
  released.

  The main updates since version 0.8.3, which was the last release
  before Romano started co-maintaining the project, are the following
  --- part coded by Romano, part by several collaborators around the
  internet:

  \begin{itemize}
  \tightlist
  \item
    The manual has been reorganized and extended, with the addition of a
    tutorial part; tens of examples have been added all over the map.
  \item
    Around 74 new shapes where added. Notably, now there are chips,
    mux-demuxes, multi-terminal transistors, several types of switches,
    flip-flops, vacuum tubes, 7-segment displays, more amplifiers, and
    so on.
  \item
    Several existing shapes have been enhanced; for example, logic gates
    have a variable number of inputs, transistors are more configurable,
    resistors can be shaped more, and more.
  \item
    You can style your circuit, changing relative sizes, default
    thickness and fill color, and more details of how you like your
    circuit to look; the same you can do with labels (voltages,
    currents, names of components and so on).
  \item
    A lot of bugs have been squashed; especially the (very complex)
    voltage direction conundrum has been clarified and you can choose
    your preferred style here too.
  \end{itemize}
\end{itemize}

A detailed list of changes can be seen below.

\begin{itemize}
\item
  Version 1.0.0-pre3 (not released)

  \begin{itemize}
  \tightlist
  \item
    Added a Reed switch
  \item
    Put the copyright and license notices on all files and update them
  \item
    Fixed the loading of style; we should not guard against reload
  \end{itemize}
\item
  Version 1.0.0-pre2 (2020-01-23)

  \textbf{Really} last additions toward the 1.0.0 version. The most
  important change is the addition of multiplexer and de-multiplexers;
  also added the multi-wires (bus) markers.

  \begin{itemize}
  \tightlist
  \item
    Added mux-demux shapes
  \item
    Added the possibility to suppress the input leads in logic gates
  \item
    Added multiple wires markers
  \item
    Added a style to switch off the automatic rotation of instruments
  \item
    Changed the shape of the or-type american logic ports (reversible
    with a flag)
  \end{itemize}
\item
  Version 1.0.0-pre1 (2019-12-22)

  Last additions before the long promised 1.0! In this pre-release we
  feature a flip-flop library, a revamped configurability of amplifiers
  (and a new amplifier as a bonus) and some bug fix around the clock.

  \begin{itemize}
  \tightlist
  \item
    Added a flip-flop library
  \item
    Added a single-input generic amplifier with the same dimension as
    ``plain amp''
  \item
    Added border anchors to amplifiers
  \item
    Added the possibility (expert only!) to add transparency to poles
    (after a suggestion from user @matthuszagh on GitHub)
  \item
    Make plus and minus symbol on amplifiers configurable
  \item
    Adjusted the position of text in triangular amplifiers
  \item
    Fixed ``plain amp'' not respecting ``noinv input up''
  \item
    Fixed minor incompatibility with ConTeXt and Plain TeX
  \end{itemize}
\item
  Version 0.9.7 (2019-12-01)

  The important thing in this release is the new position of
  transistor's labels; see the manual for details.

  \begin{itemize}
  \tightlist
  \item
    Fix the position of transistor's text. There is an option to revert
    to the old behavior.
  \item
    Added anchors for adding circuits (like snubbers) to the flyback
    diodes in transistors (after a suggestion from @EdAlvesSilva on
    GitHub).
  \end{itemize}
\item
  Version 0.9.6 (2019-11-09)

  The highlights of this release are the new multiple terminals BJTs and
  several stylistic addition and fixes; if you like to pixel-peep, you
  will like the fixed transistors arrows. Additionally, the transformers
  are much more configurable now, the ``pmos'' and ``nmos'' elements
  have grown an optional bulk connection, and you can use the ``flow''
  arrows outside of a path.

  Several small and less small bugs have been fixed.

  \begin{itemize}
  \tightlist
  \item
    Added multi-collectors and multi-emitter bipolar transistors
  \item
    Added the possibility to style each one of the two coils in a
    transformer independently
  \item
    Added bulk connection to normal MOSFETs and the respective anchors
  \item
    Added ``text'' anchor to the flow arrows, to use them alone in a
    consistent way
  \item
    Fixed flow, voltage, and current arrow positioning when ``auto'' is
    active on the path
  \item
    Fixed transistors arrows overshooting the connection point, added a
    couple of anchors
  \item
    Fixed a spelling error on op-amp key ``noinv input down''
  \item
    Fixed a problem with ``quadpoles style=inner'' and ``transformer
    core'' having the core lines running too near
  \end{itemize}
\item
  Version 0.9.5 (2019-10-12)

  This release basically add features to better control labels, voltages
  and similar text ``ornaments'' on bipoles, plus some other minor
  things.

  On the bug fixes side, a big incompatibility with ConTeXt has been
  fixed, thanks to help from \texttt{@TheTeXnician} and \texttt{@hmenke}
  on \texttt{github.com}.

  \begin{itemize}
  \tightlist
  \item
    Added a ``midtap'' anchor for coils and exposed the inner coils
    shapes in the transformers
  \item
    Added a ``curved capacitor'' with polarity coherent with
    ``ecapacitor''
  \item
    Added the possibility to apply style and access the nodes of
    bipole's text ornaments (labels, annotations, voltages, currents and
    flows)
  \item
    Added the possibility to move the wiper in resistive potentiometers
  \item
    Added a command to load and set a style in one go
  \item
    Fixed internal font changing commands for compatibility with ConTeXt
  \item
    Fixed hardcoded black color in ``elko'' and ``elmech''
  \end{itemize}
\item
  Version 0.9.4 (2019-08-30)

  This release introduces two changes: a big one, which is the styling
  of the components (please look at the manual for details) and a change
  to how voltage labels and arrows are positioned. This one should be
  backward compatible \emph{unless} you used \texttt{voltage\ shift}
  introduced in 0.9.0, which was broken when using the global
  \texttt{scale} parameter.

  The styling additions are quite big, and, although in principle they
  are backward compatible, you can find corner cases where they are not,
  especially if you used to change parameters for
  \texttt{pgfcirc.defines.tex}; so a snapshot for the 0.9.3 version is
  available.

  \begin{itemize}
  \tightlist
  \item
    Fixed a bug with ``inline'' gyrators, now the circle will not
    overlap
  \item
    Fixed a bug in input anchors of european not ports
  \item
    Fixed ``tlinestub'' so that it has the same default size than
    ``tline'' (TL)
  \item
    Fixed the ``transistor arrows at end'' feature, added to styling
  \item
    Changed the behavior of ``voltage shift'' and voltage label
    positioning to be more robust
  \item
    Added several new anchors for ``elmech'' element
  \item
    Several minor fixes in some component drawings to allow fill and
    thickness styles
  \item
    Add 0.9.3 version snapshots.
  \item
    Added styling of relative size of components (at a global or local
    level)
  \item
    Added styling for fill color and thickeness
  \item
    Added style files
  \end{itemize}
\item
  Version 0.9.3 (2019-07-13)

  \begin{itemize}
  \tightlist
  \item
    Added the option to have ``dotless'' P-MOS (to use with arrowmos
    option)
  \item
    Fixed a (puzzling) problem with coupler2
  \item
    Fixed a compatibility problem with newer PGF (\textgreater{}3.0.1a)
  \end{itemize}
\item
  Version 0.9.2 (2019-06-21)

  \begin{itemize}
  \tightlist
  \item
    (hopefully) fixed ConTeXt compatibility. Most new functionality is
    not tested; testers and developers for the ConTeXt side are needed.
  \item
    Added old ConTeXt version for 0.8.3
  \item
    Added tailless ground
  \end{itemize}
\item
  Version 0.9.1 (2019-06-16)

  \begin{itemize}
  \tightlist
  \item
    Added old LaTeX versions for 0.8.3, 0.7, 0.6 and 0.4
  \item
    Added the option to have inline transformers and gyrators
  \item
    Added rotary switches
  \item
    Added more configurable bipole nodes (connectors) and more shapes
  \item
    Added 7-segment displays
  \item
    Added vacuum tubes by J. op den Brouw
  \item
    Made the open shape of dcisources configurable
  \item
    Made the arrows on vcc and vee configurable
  \item
    Fixed anchors of diamondpole nodes
  \item
    Fixed a bug (\#205) about unstable anchors in the chip components
  \item
    Fixed a regression in label placement for some values of scaling
  \item
    Fixed problems with cute switches anchors
  \end{itemize}
\item
  Version 0.9.0 (2019-05-10)

  \begin{itemize}
  \tightlist
  \item
    Added Romano Giannetti as contributor
  \item
    Added a CONTRIBUTING file
  \item
    Added options for solving the voltage direction problems.
  \item
    Adjusted ground symbols to better match ISO standard, added new
    symbols
  \item
    Added new sources (cute european versions, noise sources)
  \item
    Added new types of amplifiers, and option to flip inputs and outputs
  \item
    Added bidirectional diodes (diac) thanks to Andre Lucas Chinazzo
  \item
    Added L,R,C sensors (with european, american and cute variants)
  \item
    Added stacked labels (thanks to the original work by Claudio
    Fiandrino)
  \item
    Make the position of voltage symbols adjustable
  \item
    Make the position of arrows in FETs and BJTs adjustable
  \item
    Added chips (DIP, QFP) with a generic number of pins
  \item
    Added special anchors for transformers (and fixed the wrong center
    anchor)
  \item
    Changed the logical port implementation to multiple inputs (thanks
    to John Kormylo) with border anchors.
  \item
    Added several symbols: bulb, new switches, new antennas,
    loudspeaker, microphone, coaxial connector, viscoelastic element
  \item
    Make most components fillable
  \item
    Added the oscilloscope component and several new instruments
  \item
    Added viscoelastic element
  \item
    Added a manual section on how to define new components
  \item
    Fixed american voltage symbols and allow to customize them
  \item
    Fixed placement of straightlabels in several cases
  \item
    Fixed a bug about straightlabels (thanks to @fotesan)
  \item
    Fixed labels spacing so that they are independent on scale factor
  \item
    Fixed the position of text labels in amplifiers
  \end{itemize}
\item
  Version 0.8.3 (2017-05-28)

  \begin{itemize}
  \tightlist
  \item
    Removed unwanted lines at to-paths if the starting point is a node
    without a explicit anchor.
  \item
    Fixed scaling option, now all parts are scaled by bipoles/length
  \item
    Surge arrester appears no more if a to path is used without
    {[}{]}-options
  \item
    Fixed current placement now possible with paths at an angle of
    around 280°
  \item
    Fixed voltage placement now possible with paths at an angle of
    around 280°
  \item
    Fixed label and annotation placement (at some angles position not
    changable)
  \item
    Adjustable default distance for straight-voltages:
    `bipoles/voltage/straight label distance'
  \item
    Added Symbol for bandstop filter
  \item
    New annotation type to show flows using f=\ldots{} like currents,
    can be used for thermal, power or current flows
  \end{itemize}
\item
  Version 0.8.2 (2017-05-01)

  \begin{itemize}
  \tightlist
  \item
    Fixes pgfkeys error using alternatively specified mixed colors(see
    pgfplots manual section ``4.7.5 Colors'')
  \item
    Added new switches ``ncs'' and ``nos''
  \item
    Reworked arrows at spst-switches
  \item
    Fixed direction of controlled american voltage source
  \item
    ``v\textless{}='' and ``i\textless{}='' do not rotate the sources
    anymore(see them as ``counting direction indication'', this can be
    different then the shape orientation); Use the option ``invert'' to
    change the direction of the source/apperance of the shape.
  \item
    current label ``i='' can now be used independent of the regular
    label ``l='' at current sources
  \item
    rewrite of current arrow placement. Current arrows can now also be
    rotated on zero-length paths
  \item
    New DIN/EN compliant operational amplifier symbol ``en amp''
  \end{itemize}
\item
  Version 0.8.1 (2017-03-25)

  \begin{itemize}
  \tightlist
  \item
    Fixed unwanted line through components if target coordinate is a
    name of a node
  \item
    Fixed position of labels with subscript letters.
  \item
    Absolute distance calculation in terms of ex at rotated labels
  \item
    Fixed label for transistor paths (no label drawn)
  \end{itemize}
\item
  Version 0.8 (2017-03-08)

  \begin{itemize}
  \tightlist
  \item
    Allow use of voltage label at a {[}short{]}
  \item
    Correct line joins between path components (to{[}\ldots{}{]})
  \item
    New Pole-shape .-. to fill perpendicular joins
  \item
    Fixed direction of controlled american current source
  \item
    Fixed incorrect scaling of magnetron
  \item
    Fixed: Number of american inductor coils not adjustable
  \item
    Fixed Battery Symbols and added new battery2 symbol
  \item
    Added non-inverting Schmitttrigger
  \end{itemize}
\item
  Version 0.7 (2016-09-08)

  \begin{itemize}
  \tightlist
  \item
    Added second annotation label, showing, e.g., the value of an
    component
  \item
    Added new symbol: magnetron
  \item
    Fixed name conflict of diamond shape with tikz.shapes package
  \item
    Fixed varcap symbol at small scalings
  \item
    New packet-option ``straightvoltages, to draw straight(no curved)
    voltage arrows
  \item
    New option ``invert'' to revert the node direction at paths
  \item
    Fixed american voltage label at special sources and battery
  \item
    Fixed/rotated battery symbol(longer lines by default positive
    voltage)
  \item
    New symbol Schmitttrigger
  \end{itemize}
\item
  Version 0.6 (2016-06-06)

  \begin{itemize}
  \tightlist
  \item
    Added Mechanical Symbols (damper,mass,spring)
  \item
    Added new connection style diamond, use (d-d)
  \item
    Added new sources voosource and ioosource (double zero-style)
  \item
    All diode can now drawn in a stroked way, just use globel option
    ``strokediode'' or stroke instead of full/empty, or D-. Use this
    option for compliance with DIN standard EN-60617
  \item
    Improved Shape of Diodes:tunnel diode, Zener diode, schottky diode
    (bit longer lines at cathode)
  \item
    Reworked igbt: New anchors G,gate and new L-shaped form Lnigbt,
    Lpigbt
  \item
    Improved shape of all fet-transistors and mirrored p-chan fets as
    default, as pnp, pmos, pfet are already. This means a
    backward-incompatibility, but smaller code, because p-channels
    mosfet are by default in the correct direction(source at top). Just
    remove the `yscale=-1' from your p-chan fets at old pictures.
  \end{itemize}
\item
  Version 0.5 (2016-04-24)

  \begin{itemize}
  \tightlist
  \item
    new option boxed and dashed for hf-symbols
  \item
    new option solderdot to enable/disable solderdot at source port of
    some fets
  \item
    new parts: photovoltaic source, piezo crystal, electrolytic
    capacitor, electromechanical device(motor, generator)
  \item
    corrected voltage and current direction(option to use old behaviour)
  \item
    option to show body diode at fet transistors
  \end{itemize}
\item
  Version 0.4

  \begin{itemize}
  \tightlist
  \item
    minor improvements to documentation
  \item
    comply with TDS
  \item
    merge high frequency symbols by Stefan Erhardt
  \item
    added switch (not opening nor closing)
  \item
    added solder dot in some transistors
  \item
    improved ConTeXt compatibility
  \end{itemize}
\item
  Version 0.3.1

  \begin{itemize}
  \tightlist
  \item
    different management of color\ldots{}
  \item
    fixed typo in documentation
  \item
    fixed an error in the angle computation in voltage and current
    routines
  \item
    fixed problem with label size when scaling a tikz picture
  \item
    added gas filled surge arrester
  \item
    added compatibility option to work with Tikz's own circuit library
  \item
    fixed infinite in arctan computation
  \end{itemize}
\item
  Version 0.3.0

  \begin{itemize}
  \tightlist
  \item
    fixed gate node for a few transistors
  \item
    added mixer
  \item
    added fully differential op amp (by Kristofer M. Monisit)
  \item
    now general settings for the drawing of voltage can be overridden
    for specific components
  \item
    made arrows more homogeneous (either the current one, or latex' bt
    pgf)
  \item
    added the single battery cell
  \item
    added fuse and asymmetric fuse
  \item
    added toggle switch
  \item
    added varistor, photoresistor, thermocouple, push button
  \item
    added thermistor, thermistor ptc, thermistor ptc
  \item
    fixed misalignment of voltage label in vertical bipoles with names
  \item
    added isfet
  \item
    added noiseless, protective, chassis, signal and reference grounds
    (Luigi «Liverpool»)
  \end{itemize}
\item
  Version 0.2.4

  \begin{itemize}
  \tightlist
  \item
    added square voltage source (contributed by Alistair Kwan)
  \item
    added buffer and plain amplifier (contributed by Danilo Piazzalunga)
  \item
    added squid and barrier (contributed by Cor Molenaar)
  \item
    added antenna and transmission line symbols contributed by Leonardo
    Azzinnari
  \item
    added the changeover switch spdt (suggestion of Fabio Maria
    Antoniali)
  \item
    rename of context.tex and context.pdf (thanks to Karl Berry)
  \item
    updated the email address
  \item
    in documentation, fixed wrong (non-standard) labelling of the axis
    in an example (thanks to prof. Claudio Beccaria)
  \item
    fixed scaling inconsistencies in quadrupoles
  \item
    fixed division by zero error on certain vertical paths
  \item
    introduced options straighlabels, rotatelabels, smartlabels
  \end{itemize}
\item
  Version 0.2.3

  \begin{itemize}
  \tightlist
  \item
    fixed compatibility problem with label option from tikz
  \item
    Fixed resizing problem for shape ground
  \item
    Variable capacitor
  \item
    polarized capacitor
  \item
    ConTeXt support (read the manual!)
  \item
    nfet, nigfete, nigfetd, pfet, pigfete, pigfetd (contribution of
    Clemens Helfmeier and Theodor Borsche)
  \item
    njfet, pjfet (contribution of Danilo Piazzalunga)
  \item
    pigbt, nigbt
  \item
    \emph{backward incompatibility} potentiometer is now the standard
    resistor-with-arrow-in-the-middle; the old potentiometer is now
    known as variable resistor (or vR), similarly to variable inductor
    and variable capacitor
  \item
    triac, thyristor, memristor
  \item
    new property ``name'' for bipoles
  \item
    fixed voltage problem for batteries in american voltage mode
  \item
    european logic gates
  \item
    \emph{backward incompatibility} new american standard inductor. Old
    american inductor now called ``cute inductor''
  \item
    \emph{backward incompatibility} transformer now linked with the
    chosen type of inductor, and version with core, too. Similarly for
    variable inductor
  \item
    \emph{backward incompatibility} styles for selecting shape variants
    now end are in the plural to avoid conflict with paths
  \item
    new placing option for some tripoles (mostly transistors)
  \item
    mirror path style
  \end{itemize}
\item
  Version 0.2.2 - 20090520

  \begin{itemize}
  \tightlist
  \item
    Added the shape for lamps.
  \item
    Added options \texttt{europeanresistor}, \texttt{europeaninductor},
    \texttt{americanresistor} and \texttt{americaninductor}, with
    corresponding styles.
  \item
    FIXED: error in transistor arrow positioning and direction under
    negative \texttt{xscale} and \texttt{yscale}.
  \end{itemize}
\item
  Version 0.2.1 - 20090503

  \begin{itemize}
  \tightlist
  \item
    Op-amps added
  \item
    added options arrowmos and noarrowmos, to add arrows to pmos and
    nmos
  \end{itemize}
\item
  Version 0.2 - 20090417 First public release on CTAN

  \begin{itemize}
  \tightlist
  \item
    \emph{Backward incompatibility}: labels ending with
    \texttt{:}\textit{angle} are not parsed for positioning anymore.
  \item
    Full use of \TikZ~keyval features.
  \item
    White background is not filled anymore: now the network can be drawn
    on a background picture as well.
  \item
    Several new components added (logical ports, transistors, double
    bipoles, \ldots).
  \item
    Color support.
  \item
    Integration with \{\ttfamily siunitx\}.
  \item
    Voltage, american style.
  \item
    Better code, perhaps. General cleanup at the very least.
  \end{itemize}
\item
  Version 0.1 - 2007-10-29 First public release
\end{itemize}
