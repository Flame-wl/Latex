%告知 CTeX 这个文件是用 UTF-8 编码
% !Mode:: "TeX:UTF-8"

%使用 pdflatex
%!TEX program = pdflatex


\documentclass[12pt,a4paper]{article}
%正文字体大小为12pt, 页面规格是A4, 使用article文档格式
\usepackage[utf8]{inputenc}
%作用是inputenc用来识别输入编码
\usepackage[left=2.2cm,right=2.2cm,top=3cm,bottom=2.5cm]{geometry}
%latex设置页面边距,页面大小,页边距
\usepackage{mathtools}
%数学公式扩展宏包,提供了公式编号定制和更多的符号、矩阵等
\usepackage{booktabs}  
%booktabs宏包画三线表,线条精细可变
\usepackage{graphicx} 
%支持插图
\usepackage{listings}
%提供了排版关键字高亮的代码环境 lslisting 以及对版式的自定义。
%类似宏包有minted

\lstset{%使用\lstset{}进行代码环境的设置
backgroundcolor=\color{cyan!10},%% 选择代码背景,必须加上\ usepackage {color}或\ usepackage {xcolor}.
basicstyle=\ttfamily, % 设置代码字号.
numbers=left,% 给代码添加行号,可取值none, left, right.
numberstyle=\scriptsize %小六号 \scriptsize
% numberstyle=\tiny\color{mygray}, % 行号的字号和颜色
}

\usepackage{fancyhdr}
%修改页眉页脚格式,令页眉页脚可以左对齐、居中、右对齐

\usepackage{tikz}%以 TikZ 为基础提供排版样式丰富的彩色盒子的功能
\usepackage[europeanresistors,americaninductors]{circuitikz}
%欧式电阻,美式电感

\usepackage{indentfirst} %令章节标题后的第一段首行缩进
\usepackage[wby]{callouts}

\lstdefinestyle{mystyle}{
    backgroundcolor=\color{white},  %背景颜色 
    commentstyle=\color{codegreen}, %注释风格
    keywordstyle=\color{magenta},   %关键字风格 红紫色
    numberstyle=\tiny\color{codegray},
    stringstyle=\color{codepurple},
    basicstyle=\footnotesize\ttfamily,breaklines=true,
    %设置代码的大小  选择一种等宽(“打字机”)字体族  对过长的代码自动换行  
    breakatwhitespace=false, %空格中断
    xleftmargin=20pt, %x左边框
    xrightmargin=20pt,%x右边框       
    breaklines=true,  %代码过长则换行               
    captionpos=b,     % 设置标题位置.               
    keepspaces=true,  % 保留空格    有助于保持代码的缩进 possibly needs columns=flexible           
    numbers=left,     % 给代码添加行号,可取值none, left, right.                   
    numbersep=5pt,    % 设置行号与代码之间的间隔              
    showspaces=false, % 显示每个地方添加特定下划线的空格; 覆盖了'showtringspaces'               
    showstringspaces=false, % 仅在字符串中允许空格
    showtabs=false,   % 在字符串中显示添加特定下划线的制表符               
    tabsize=2,        % 将默认tab设置为2个空格  
    framextopmargin=50pt,%代码区定框
    frame=bottomline,   %代码区底部
    basicstyle=\footnotesize\ttfamily,  % 设置代码字号
    language=Octave     % 使用的语言
}
\usepackage{ulem}%提供排版可断行下划线的命令 \uline 以及其它装饰文字的命令
\lstset{style=mystyle} %代码环境设置  自定义版式,将mystyle中版式导入
\linespread{1.5}       %行距1.5倍 
\title{\textbf{\texttt{Electric Circuits - Homework 01}}}
\author{Automation Class 1802}
\pagestyle{fancy}   %使用fancy风格
\fancyhf{} % 清空当前设置
\rhead{HW2 Edition} %页眉右边
\rfoot{fireowl}       %页脚右边
\lhead{Electric Circuits} % 页眉左边
\cfoot{\thepage}    %页脚中间 页码
\thispagestyle{plain}
% empty
% 无页眉页脚
% plain
% 无页眉,页脚为居中页码
% headings
% 页眉为章节标题,无页脚
% myheadings
% 页眉内容可自定义,无页脚
\date{(Due date: 2020/10/5)}%自定义日期\today显示电脑上的日期-英文版


\begin{document}
\maketitle

%section
{\large This assignment covers \uwave{Ch3 and Ch4.1-4.9} of the textbook. The full credit is \uwave{100 points}. For each question, \uwave{detailed derivation processes} and \uwave{accurate numbers} are required to get full credit.}

\begin{enumerate}
    \item (10 points) \uline{Problem 3.8} of the textbook (p100), while the right resistor is change from 6 $\Omega$ to 9 $\Omega$.%欧姆符号要用美元符号
    %item 自动换行
    %\par%或者两个回车,表示换行


    ans:\\%强制分行:\\ 或 \\*[和下行间距离]
    \begin{quote}

    \begin{center}
        \begin{circuitikz}[american voltages, american currents, american resistors]
            \draw (0, 0) to[short, *- ] (-3,0) to[V, -*, l^=$300V$] (-3, 3) to [R, -*, l^=R, f_=$i_e$] (0, 3) to[-, R, l=$170\Omega$, f=$i_a$](0,0) to[R, -, l=$18\Omega$] (3, 0) to[R, -*, l^=$12\Omega$, f<_=$i_d$, v<=$60 V$] (3, 3) to[-, short] (3, 4.5) to[R, -, l=$50\Omega$, f<_=$i_b$] (-3, 4.5) to (-3, 3);
            \draw (0, 3) to[R, -, l=$10\Omega$, f=$i_c$] (3,3);
            \node at (-3.2, 3){a};
            \node at (0, 3.3){b};
            \node at (3.2, 3){c};
            \node at (0, -0.3){d};
        \end{circuitikz} 
    \end{center}

    \end{quote}

    
    \item (10 points) \uline{Problem 3.60} of the textbook (p107), while the voltage source is changed from 500 V to 900 V and the right resistor is changed from 27 $\Omega$ to 17 $\Omega$.
    
    
    \item (10 points) \uline{Problem 3.71} of the textbook (p109).
    
    
    \item (15 points) \uline{Problem 4.27} of the textbook (p155), while the voltage source is changed from 24 V to 18 V and the voltage-controlled voltage source is changed for 5v${_\Delta}$ to 3v${_\Delta}$. Also calculate v$_0$ when the 33-$\omega$ resistor is climinated.
    \item (20 points) \uline{Problem 4.38} of the textbook (p156), while the voltage source is changed from 135 V to 225 V. Also find the power extracted or dissipated by the current con-trolled voltage source.
    \item (10 points) \uline{Problem 4.45} of the textbook (p157), while the current source is changed from 20 A to 160 A and the current-controlled voltage source is changed from 6.5i${_\Delta}$ to 8i${_\Delta}$.
    \item (10 points) \uline{Problem 4.58} of the textbook (p158), while the top current source is changed from 4 A to 10 A.
    \item (10 points) \uline{Problem 4.59} of the textbook (p159), while the right current source is changed from 0.6 mA to 1.2 mA.



\end{enumerate}
\end{document}